\documentclass[margin,line]{res}
\usepackage{pgf,interval}

\usepackage[pyfuture=none]{pythontex}

\oddsidemargin -.5in
\evensidemargin -.5in
\textwidth=6.0in
\itemsep=0in
\parsep=0in
% if using pdflatex:
%\setlength{\pdfpagewidth}{\paperwidth}
%\setlength{\pdfpageheight}{\paperheight}

\newenvironment{list1}{
  \begin{list}{\ding{113}}{%
      \setlength{\itemsep}{0in}
      \setlength{\parsep}{0in} \setlength{\parskip}{0in}
      \setlength{\topsep}{0in} \setlength{\partopsep}{0in}
      \setlength{\leftmargin}{0.17in}}}{\end{list}}
\newenvironment{list2}{
  \begin{list}{$\bullet$}{%
      \setlength{\itemsep}{0in}
      \setlength{\parsep}{0in} \setlength{\parskip}{0in}
      \setlength{\topsep}{0in} \setlength{\partopsep}{0in}
      \setlength{\leftmargin}{0.2in}}}{\end{list}}


\begin{document}

\name{Seokhwan Kim \vspace*{.1in}}

\begin{resume}
\section{\sc Contact Information}
\vspace{.05in}
\begin{pycode}
import sqlite3

db = sqlite3.connect('../profile.db')
c = db.cursor()

contact_info = {}
c.execute('SELECT field, content FROM contact')
for row in c:
  contact_info[row[0]] = row[1].replace('#', '\#')
print(r'\begin{tabular}{@{}p{3in}p{4in}}')
# print(r'%s & {\it Office:} %s \\' % (contact_info['department'], contact_info['phone']))
print(r'%s & {\it Mobile:} %s \\' % (contact_info['affiliation'], contact_info['mobile']))
print(r'%s & {\it E-mail:} %s \\' % (contact_info['address1'], contact_info['email_txt']))
print(r'%s & {\it WWW:} %s \\' % (contact_info['address2'], contact_info['website']))
print(r'\end{tabular}')
\end{pycode}


\section{\sc Research Interests}
\begin{pycode}
import sqlite3

db = sqlite3.connect('../profile.db')
c = db.cursor()

kwd_list = []
c.execute('SELECT keyword FROM research_interest')
for row in c:
  kwd_list.append(row[0])
print(', '.join(kwd_list))
\end{pycode}

\section{\sc Education}
{\bf Pohang University of Science and Technology (POSTECH)}, Pohang, Korea\\
\vspace*{-.3cm}
\begin{list1}
\item[] {\em Ph.D., Computer Science and Engineering} \hfill {\em February 2012}\\
  \vspace*{-.3cm}
  \begin{list1}
  \item [] Dissertation Topic:  ``Cross-Lingual Weakly-Supervised Learning of Semantic Relations''
  \item [] Advisor:  Gary Geunbae Lee
  \item [] Committee: Jong-hyeok Lee, Seung-jin Choi, Seung-won Hwang, Sung-Hyon Myaeng
%  \item [] GPA: 3.81/4.3
  \end{list1}
  \vspace*{.3cm}
\item[] {\em B.S., Computer Science and Engineering} \hfill {\em August 2005}\\
  \vspace*{-.3cm}
  % \begin{list1}
  % \item[] GPA: 3.46/4.3 (with honors)
  % \end{list1}
\end{list1}

\section{\sc Research Experience}
{\bf Amazon Alexa AI}, Sunnyvale, CA, USA\\
\vspace{-.3cm}
\begin{list1}
\item[] {\em Sr. Machine Learning Scientist} \hfill {\em June 2019 -}\\
  \vspace{-.3cm}
\end{list1}

{\bf Adobe Research}, San Jose, CA, USA\\
\vspace{-.3cm}
\begin{list1}
\item[] {\em Research Scientist} \hfill {\em July 2017 - May 2019}\\
  \vspace{-.3cm}
\end{list1}

{\bf Institute for Infocomm Research (I2R)}, Singapore\\
\vspace{-.3cm}
\begin{list1}
\item[] {\em Scientist} \hfill {\em January 2012 - July 2017}\\
  \vspace{-.3cm}
%   \begin{list1}
%   \item[] {\bf SARA project}, Developed a conversational agent for touristic and conference information and released it in the official mobile app for Interspeech 2014.
%   \item[] {\bf COLDAR project}, Developed components for multi-lingual natural language understanding and dialog management.
%   \item[] {\bf TIRS project}, Developed components for natural language understanding and database query generation.
%   \end{list1}
\end{list1}

{\bf Max Planck Institute for Informatics (MPI-INF)}, Saarbrucken, Germany\\
\vspace{-.3cm}
\begin{list1}
\item[] {\em Research Intern} \hfill {\em April 2011 - August 2011}\\
  {\em Advisor: Prof. Gerhard Weikum} \\
  \vspace{-.3cm}
%   \begin{list1}
%     \item[] {\bf DIDO project}, Developed a higher-order noun phrase chunker for recognizing bio-medical entities with environmental contexts.
%   \end{list1}
\end{list1}

{\bf Pohang University of Science and Technology (POSTECH)}, Pohang, Korea\\
\vspace{-.3cm}
\begin{list1}
\item[] {\em Research Assistant} \hfill {\em September 2005 - December 2011}\\
  {\em Advisor: Prof. Gary Geunbae Lee} \\
  \vspace{-.3cm}
%   \begin{list1}
%     \item[] {\bf Development of Dialog-based Question Answering System for TV contents}, LG Electronics. Developed components for robust natural language understanding.
%     \item[] {\bf Development of Content Recommendation System for Digital TV}, Samsung Electronics. Developed an automatic content feeding component for contentbased recommendation algorithm.
%     \item[] {\bf Semi-supervised Information Extraction for Automatic Content Feeding in Information Access Dialog Systems}, Microsoft Research Asia. Developed an algorithm of semi-supervised information extraction.
%     \item[] {\bf Implementation of Dialog System Components to Upgrade Chatbot}, SK Telecom Co., Ltd. Developed components for natural language understanding and database management.
%     \item[] {\bf Development of Natural Language Dialog System for Database Search and Entertainment}, KT Corporation. Developed named-entity recognizer for natural language understanding.
%   \end{list1}
 \end{list1}

\section{\sc Teaching Experience}
{\bf Pohang University of Science and Technology (POSTECH)}, Pohang, Korea\\
\vspace{-.3cm}
\begin{list1}
\item [] {\em Teaching Assistant} \hfill {\em Spring 2010}\\
  \vspace{-.3cm}
  \begin{list1}
  \item[] {\bf CS101, Introduction to Computing}
  \end{list1}
  \vspace{.3cm}
\item[] {\em Teaching Assistant} \hfill {\em Spring 2006}\\
  \vspace{-.3cm}
  \begin{list1}
  \item[] {\bf CS421, Database System}
  \end{list1}
\end{list1}

\section{\sc Professional Experience}
\begin{pycode}
import sqlite3

db = sqlite3.connect('../profile.db')
c = db.cursor()

kwd_list = []
c.execute('SELECT position, org, start_year, end_year FROM ACTIVITY ORDER BY start_year DESC')
for row in c:
  position, org, start_year, end_year = row
  if start_year is not None:
    print(r'{\bf %s} \\' % (org,))
    print(r'\vspace{-.3cm}')
    print(r'\begin{list1}')
    print(r'\item[] {\em %s} \hfill {\em %d}' % (position, start_year))
    print(r'\end{list1}')

review_list = []
c.execute('SELECT org FROM ACTIVITY WHERE position = "Reviewer" ORDER BY org')
for row in c:
  review_list.append(row[0])

if len(review_list) > 0:
  print(r'{\bf %s} \\' % (', '.join(review_list),))
  print(r'\vspace{-.3cm}')
  print(r'\begin{list1}')
  print(r'\item[] {\em Reviewer}')
  print(r'\end{list1}')
\end{pycode}

\section{\sc Talks}
{\bf Tutorials}\\
\begin{pycode}
import sqlite3
import datetime

print(r'\vspace{-.3cm}')
print(r'\begin{list1}')

db = sqlite3.connect('../profile.db')
c = db.cursor()

c.execute('SELECT type, venue, venue_link, venue_abbr, date, title, slide_link, speakers FROM talk ORDER BY date DESC')
for row in c:
  talk_obj = {}

  talk_type, venue, venue_link, venue_abbr, date_str, title, slide_link, speakers = row
  talk_date = datetime.datetime.strptime(date_str, '%Y-%m-%d')

  year = talk_date.year
  month = talk_date.month
  day = talk_date.day

  if venue_abbr is not None and len(venue_abbr) > 0:
    print(r'\item[] {\em %s.} {\bf %s.} %s @ %s (%s), %04d-%02d-%02d.\\' % (', '.join(speakers.split('|')), title, talk_type, venue, venue_abbr, year, month, day))
  else:
    print(r'\item[] {\em %s.} {\bf %s.} %s @ %s, %04d-%02d-%02d.\\' % (', '.join(speakers.split('|')), title, talk_type, venue, year, month, day))
print(r'\end{list1}')
\end{pycode}



\section{\sc Publications}
{\bf International Journals}\\

\begin{pycode}
print(r'\vspace{-.3cm}')
print(r'\begin{list1}')

import sqlite3, datetime, re

db = sqlite3.connect('../profile.db')
c = db.cursor()

c.execute('SELECT title, authors, journal, volume, pages, month, year, publisher_link, pdf_link, bib_link, status FROM journal_paper WHERE locale = "international" ORDER BY year DESC, month DESC')
for row in c:
  title, authors, journal, volume, pages, month, year, publisher_link, pdf_link, bib_link, status = row
  if month is not None:
    m = re.match('^([0-9]+)-([0-9]+)$', month)
    if m:
      dateobj = datetime.date(2000, int(m.group(1)), 1)
      mon1 = dateobj.strftime('%b')
      dateobj = datetime.date(2000, int(m.group(2)), 1)
      mon2 = dateobj.strftime('%b')
      month = ', %s-%s' % (mon1, mon2)
    else:
      m = re.match('^[0-9]+$', month)
      if m:
        dateobj = datetime.date(2000, int(month), 1)
        month = dateobj.strftime('%b')

  if status is not None:
    print(r'\item[] {\em %s.} {\bf %s.} %s, (%s), %d.\\' % (', '.join(authors.split('|')), title, journal, status, year))
  else:
    print(r'\item[] {\em %s.} {\bf %s.} %s, %s (%s), %s %d.\\' % (', '.join(authors.split('|')), title, journal, volume, pages, month, year))

print(r'\end{list1}')
\end{pycode}

{\bf International Conferences}\\

\begin{pycode}
print(r'\vspace{-.3cm}')
print(r'\begin{list1}')

import sqlite3, datetime, re

db = sqlite3.connect('../profile.db')
c = db.cursor()

c.execute('SELECT title, authors, conference, conference_abbr, volume, pages, month, year, city, conference_link, publisher_link, pdf_link, bib_link, slide_link, poster_link, rate, type FROM conference_paper WHERE locale = "international" ORDER BY year DESC, month DESC')
for row in c:
  title, authors, conference, conference_abbr, volume, pages, month, year, city, conference_link, publisher_link, pdf_link, bib_link, slide_link, poster_link, rate, tp = row
  src = ''
  if conference is not None:
    today = datetime.date.today()
    if year > today.year or (year == today.year and month > today.month):
      src += 'To appear in '
    src += 'Proceedings of %s' % (conference,)

  if month is not None:
    dateobj = datetime.date(2000, int(month), 1)
    month = dateobj.strftime('%b')

  print(r'\item[]')
  if authors is not None:
    print(r'{\em %s.}' % (', '.join(authors.split('|')),))
  if title is not None:
    print(r'{\bf %s.}' % (title.replace('&', '\&'),))
  if src is not None and conference_abbr is not None:
    print(r'%s (%s),' % (src, conference_abbr))
  elif src is not None:
    print(r'%s,' % (src,))
  if volume is not None and len(str(volume)) > 0:
    print(r'Vol. %s,' % (volume,))
  if pages is not None and len(pages) > 0:
    print(r'pp %s,' % (pages,))
  if city is not None:
    print(r'%s,' % (city,))
  print(r'%s %d.' % (month, year))

  if tp is not None and len(tp) > 0:
    print(r'(%s)' % (tp,))
  if rate is not None and len(str(rate)) > 0:
    print(r'(%s\%% acceptance)' % (rate,))

  print(r'\\')
print(r'\end{list1}')
\end{pycode}

% {\bf Domestic Journals and Conferences}\\
% \vspace{-.3cm}
% \begin{list1}
% \item[] I published several Korean journals and conference papers.
% \item[] Please visit http://seokhwankim.com\\
% \end{list1}

{\bf Patents}\\
\begin{pycode}
print(r'\vspace{-.3cm}')
print(r'\begin{list1}')

import sqlite3, datetime, re

db = sqlite3.connect('../profile.db')
c = db.cursor()

c.execute('SELECT title_en, authors, reg_no, reg_date, file_no, file_date, country FROM patent WHERE type = "patent" AND title_en IS NOT NULL ORDER BY DATE(reg_date) DESC')
for row in c:
  title_en, authors, reg_no, reg_date, file_no, file_date, country = row
  if title_en is not None and len(title_en) > 0:
    print(r'\item[] {\em %s.} {\bf %s.}' % (', '.join(authors.split('|')), title_en))
  if country is not None and len(country) > 0:
    print(r'Registration \#%s, %s,' % (reg_no, country))
  else:
    print(r'%s,' % (reg_no,))

  print(r'%s\\' % (reg_date,))

print(r'\end{list1}')
\end{pycode}

\section{\sc Skills}
{\bf Programming Languages}: Python, C++, Java\\
{\bf Machine Learning Toolkits}: PyTorch, TensorFlow, scikit-learn, SVM-Light, CRFsuite\\
{\bf NLP and IR Toolkits}: NLTK, spaCy, Stanford CoreNLP, Lucene\\
{\bf Databases}: MySQL, SQLite, MongoDB, Neo4j\\

\section{\sc References}
Available upon request
\end{resume}
\end{document}
